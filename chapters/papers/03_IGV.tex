\chapter{Integrative genomics viewer}

\section{Introduction}
Experienced human review is essential in analysis of the datasets generated during genomic studies.
The integrative genomics viewer or IGV is a visualization tool that enables intuitive real-time exploration of diverse, large scale genomic data sets.
It supports integration of aligned sequence reads, mutations, copy number, RNA interference screens, gene expression, methylation and genomic annotations.
IGV makes use of efficient, multi-resolution file formats to enable real-time exploration of arbitrarily large data sets over all resolution scales.
The user can zoom and pan across the genome at any level of detail, from whole genome to base pair.
Sample annotations can be defined and data divided into tracks.
Annotations are displayed as a heatmap.
Its scalable architecture makes it well suited for genome-wide exploration of NGS datasets, both basic aligned read and its derived results.
As the user zooms below the $50kb$ range individual aligned reads become visible and putative SNPs are highlighted as allele counts in the coverage plot.
Zooming in further individual base mismatches become visible, highlighted by color and intensity according to base call and quality.
Reads can be sorted by quality, strand, sample and other attributes.
IGV use paired ends reads to color-code paired ends if their insert sizes are larger than expected, fall on different chromosomes or have unexpected pair orientations.
Intra and inter chromosomal events are readily distinguished by color-coding.
