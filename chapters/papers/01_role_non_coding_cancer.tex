\chapter{Role of non-coding sequence variants in cancer}

\section{Abstract}
Patients with cancer carry somatic sequence variants in their tumour in addition to germine variants in their inherited genome.
Numerous studies have noted the importance of non-coding variants in cancer.
The overwhelming majority of variants occur in non-coding portion of the genome.

	\subsection{Introduction}
	One of the most important benefit of whole genome sequencing is the identification of variants in non-coding regions of the genome, with most of them lying in such regions.
	One of the biggest challenges is to identify driver mutation and distinguish them from passenger mutations.

\section{Genomic sequence variants}
The general properties of sequence variants are applicable to non-coding variants.
They range from single nucleotide variants to small insertion and deletion of less than $50bp$ or indels, to larger structural variants.
The latter can be copy number variants CNV or copy-number neutral.
An average human genome contains $4$ million germline sequence variants, whereas a tumour genome contains thousands of variants relative to the same individual germline DNA.
Somatic variants are rarer in healthy tissues.
Somatic mutation frequency varies across different cancer types.
Some germline variants may be responsible for tumorigenesis (high penetrance) or modulate the effect of somatic variants (low penetrance).
The germline variants associated with increased cancer susceptibility do not have a fitness effect at reproductive age, which can be the reason for the continued prevalence of such variants in the population.
Germline variants show LD that increase the difficulty in disentangling the causal disease variants.
A much higher fraction of somatic variants consist of structural variants and unlike germline variants they happen on a specific tissue.
However germline variants can have a functional effect in specific tissue if they occur in regions of closed chromatine or if they disrupt a binding site of a tissue-specific transcription factor.
Kataegis is characteristic only of somatic variants.
Moreover somatic variants are not inherited and so they are not subject to meiosis and do not show LD.

\section{Non-coding element annotation}
Non coding elements can have diverse roles in the regulation of protein-coding genes.
They consist of cis-regulatory regions and ncRNAs.
They are identified by functional genomics approaches or sequence conservation and display cell and tissue specificity.

	\subsection{Cis regulatory regions}
	Cis-regulatory regions include promoters and distal elements which regulate gene expression following binding by TFs.
	TFs bind to specific DA sequences within thier larger regions of occupancy which can be identified using chromatin immunoprecipitation followed by sequencing assays.
	They bind DNA in regions of open chromatin identified using DNAase I hypersensitivity assays and DNAase I footprinting.
	DNA methylation and other histone modification can modulate TF accessibility.
	Several histone marks are ssociated with specific putative functions.
	Most of sequence-specific TF and chromatin maks lead to highly localized ChIP-seq signals, other marks are associated with large genomic domains.
	Epigenetic hcanges can alter FT accessibility in  different cellular states and can change the activity or regulatory elements.


	\subsection{Distal regulatory elements}
	Distal regulatory elements may regulate gene expression by interacting with promoters in the 3D structure of the genome.
	Linking them to their target region is crucial to understand the effects of sequence variants in them.
	Multiple approaches have been used linke chromosome conformation capture: regulatory sequences can control transcription by looping to and physically contacting target coding genes that are located tens or hundreds of kilobases away.
	It probes one-versus-one contacts in the 3D space of the genome.
	Other variation control one-versus-all, many-versus-many and all-versus-all contacts.
	Other approches include correlation of histone marks at enhancer regions and target gene expression across multiple cell lines.
	Links between epression quantitative trait loci and associated genes.
	The resulting linkages can be studied as a comprehensive network.

	\subsection{RNA-seq}
	RNA-seq reveals non-coding transcripts, which can be confirmed to not have protein-coding ability by the absence of open reading frames or proteomic analysis.
	Certain histone modification can also indicate ncRNA activity.
	ncRNA can be divided into categories and they act through different mechanisms to modulate genexpressions.
	In particular miRNA and lncRNA are important in cancer biology.
	miRNA inhibit target gene expression by binding to the 3'-UTR and causing mRNA degradation or repression of translations.
	The mechanisms of action of lncRNA remain unclear, but a number of lncRNA have been shown to act as molecular scaffolds that bind proteins, DNA or other RNA molecules and are able to modulate gene expression.

	\subsection{Transcribed pseudogenes}
	Transcribed pseudogenes are a type of ncRNA tha bear a clear resemblance to functioning protein coding genes.
	They are copies of coding genes that have lost their ability to code for proteins owind to disabling mutations.
	They can be divided into duplicated and processed based on their formation from duplication or retrotransposition of the parent gene.
	Processed pseudogenes lack the promoter sequence and intronic strucutre and contain a 3'-poly(A) tail.
	These pseudogenes can be transcribed and regulated the expression of their parent genes, generating endo-siRNA and partecipating in the RNA interference pathway or by acting as molecular sponges.

	\subsection{Evolutionary conservation}
	Evolutionary conservation of genomic sequence across multiple species is used to annotate non-coding regions.
	Comparative analysis allowed the discovery of these ultra-conserved elements, the majority of which do not overlap protein-coding exons.
	Analysis of these sequence is important because they have been show to have a role in cancer biology.
	Non-coding elements exhibit conservation among humans.
	Negative selection within the population can be estimated using enrichment of rare alleles and reduced density of single nucleotide polymorphisms.
	These can be important to identify elements that show human-specific conservation in functional non-coding categories.
	The ultra sensitive elements and have strong depletion of common polymorphisms and enrichment of known disease-causing mutation
	Negative selection can be used to identify candidate cancer driving mutations.

\section{Roles for somatic variants in cancer}
Because cancer genomes contain a higher fraction of structural variants than germline genomes, variant detection becomes challenging.
The depth of coverage needs to be more than typically used to account for the decreased purity and increased ploidy.

	\subsection{Gain of TF-binding sites}
	TERT encodes the catalytic subunit of the enzyme telomerase.
	This allow to lenghtens telomeres, allowing cells to escape apoptosis and become cancerous.
	TERT expression is typically repressed, but it can be overexpressed in cancer.
	Recurrent mutation in the promoter of TERT in many different cancer types have been found.
	These mutations create binding motifs for the ETS family like TCF leading to their binding to TERT and upregulation of its expression.
	Tumours in tissues with low rates of self-renewal tend to exhibit higher frequencies of TERT promoter mutations.
	Gain of TF-binding site has been observed for enhancers, an important distal cis-regulatory elements that play a major part in gene transcription.

	\subsection{Fusion events due to genomic rearrangements}
	Genomic rearrangments can lead to fusion of active regulatory elements with oncogenes.
	Moreover somatic structural variants juxtapose coding sequences proximal to active enhancers during enhancer hijacking.
	So in these genomic rearrangements bring oncogenes adjacent to active promoters or enhancers.

	\subsection{ncRNAs and their binding sites}
	Disregulation of ncRNAs is a cancer signature and it can be due to the presence of somatic variants in them.
	MALATI or metastasis-associated lung adenocarcinoma transcript 1 is an example of this.
	Mutation of MALATI might be under positive selection in the tumour.
	In another example copy number amplification of a lncRNA is thought to contribute to neuroblastoma progression.
	Mutation in the binding sites of ncRNA are linked to cancer.

	\subsection{Role of pseudogenes in modulating the expression of a parental gene}
	becase of their resemblance to their parental protein-coding genes, transcribed pseudogenes are thought to have a natura way of affecting and regulating their parental counterparts.
	Pseudogene deletion or amplification can affect competition ofr miRNA binding.

\section{Roles for germline variants in cancer}
Most of the non-coding germline variants associated with cancer susceptibility can be analyzed through WGS data from healthy and ill individual.
Germline-non coding variants can affect gene expression in many different ways: point mutation can distrupt binding motifs.
GWAS SNPs and the one in LD with them might help to identify the causal variants and shed light on their mechanism of actions.

	\subsection{Promoter mutations}
	Germline mutation can create binding motifs with functional effects in the tissues where the TF is expressed.
	Moreover they can upregulate the binding.

	\subsection{SNPs in enhancers}
	Multiple SNPs in a gene desert can increase the risk of cancer: this can be due to the fact that they happens in regions that act as enhancers .
	Tissue specificity might be the reason why they are associated with specific cancers.
	Hormone-regulated cancers have mutation in TF-binding sites that vary with age owing to a differential TF activity during a person lifetime.

	\subsection{Variants in introns}
	Variants in introns can affect splice sites and cause loss of regulatory repressor elements.
	Germline CNV spanning intronic inhibitor regulatory elements can lead to the overexpression of target transcripts, modulating cell proliferation or migration.

	\subsection{SNPs in ncRNA and their binding sites}
	Most cancer-associated polymorphisms are related to increased risk, some of them can be beneficial.

	\subsection{Others}
	Other methods to identify variants with functional consequences such as ECTS and allele-specific expression analysis have been used to interpret cancer-associated loci identified through GWAS.
	These reveal germline determinants of gene expression in tumours and help to establish a link between non-coding risk loci and their target coding genes.

\section{Interplay between germline and somatic variants}
Cancer results from a complex interplay of inherited germline and acquired somatic variants.
Loss of heterozygosity events affecting non-coding element have been observed.
Somatic variants disrupt the only functioning copy of the non-coding element.
One example is the loss of miRNA or lncRNA.
However some mutation can weaken the effect of a somatic variant.

\section{Computational methods for identifying variants}
Computational prediction of drivers is a challenging task.
Driver identification uses detection of signals of positive selection or prediction of mutations with high functional impact.
Analysis of the recurrence of somatic variants from tumour samples in functional elements to identify regions under positive selection is similar to the burden test strategy.
Such analysis can be done in a specific cancer type or across multiple cancers.
In addition tools that try to do this need to account for genomic mutation rate covariats that lead to mutational heterogeneity across the genome.
Computational identification of non-coding drivers is more challenging than the coding one because of thei complex and varied modes of action.
Non-coding mutation are also more abundant and the key mutations have to be distinguished from a larger set of passenger events.
Some methods analyse the recurrence of somatic variants from tumour samples in functional elements.
Tools exits to annotate and prioritize potentially functional non-coding variants with high impact.
These tools can interpet SNV and indels or some structural variants.
Some of them try to interpret the effect of cis-regulatory mutations at a nucleotide level of resolution by computing whether they create new TF-binding motifs.
Biological networks can provide information about the connectivities of the target genes of non-coding variants.
High inter and intra-species conservation tend to be an indicator of function.

\section{Experimental approaches for functional validation}
Experimental approaches to understand the effects of cis-regulatory mutations in promoters and enhancers on cellular functions have main strategies.
First they require introducing the sequence variants,determining the resulting molecular level effects on transcription using high and low throughput functional assays and demonstrating direct biological significance.
One way to intrduce sequence variants involves the use of CRISPSR-Cas9 systems.
Then the effect evaluated through sequencing screening or luciferase reporter assays.
Analysis of the mutation in a high-throughput manner can be achieved using a modification of cis-regulatory eleemnt analysis by sequencing.
Synthetic promoter libraries drive the expression of a common reporter gene and a downstream unique barcod sequence that identifies the upstream promoter.
RNA-seq reveals the effects of promoter variants on the expression levels of their paired barcode sequence.
The activity of enhancers it independent of their location, so they can be incorporated into high-throughout reporter assays using different reporter construct arrangements.
In CRE-seq approaches the enhancer is placed upstream of the reporte gene and the barcode.
The cloned libraries can be transfected into eukaryotic cells in pooled format and RNA-seq is used to asses the resulting expression level of the reporter driven by each variatn element.
Visible reporte assays using synthetic transcription reporter construct that contain the regulatory sequence of the reporter gene enable direct validation.
Other approaches are needed to validate variants in ncRNA, UTR and intros.
Monogene assays can be used to test the effects of intronic variants: the variant sequence is cloned into transcription-competent minigene vectors and transfected into mammalian cells.
This is followed by examination of the splicing patterns of the transcripts.
Functional screening help identify the best candidates  but still needs tumour type specific validation.
Functional valdation requires demonstrating oncogenic properties that are increased owing to the variant in question.
Wild type and mutants are compared in vitro and in vivo.
Overall functional validation of non-coding variants is important to understand their biological consequence.
High-throughput prioritization of putative functional mutations is crucial before testing of the most promising candidates in in vivo systems.
