\chapter{Role of non-coding sequence variants in cancer}

\section{Abstract}
Patients with cancer carry somatic sequence variants in their tumour in addition to germine variants in their inherited genome.
Numerous studies have noted the importance of non-coding variants in cancer.
The overwhelming majority of variants occur in non-coding portion of the genome.

	\subsection{Introduction}
	One of the most important benefit of whole genome sequencing is the identification of variants in non-coding regions of the genome, with most of them lying in such regions.
	One of the biggest challenges is to identify driver mutation and distinguish them from passenger mutations.

\section{Genomic sequence variants}
The general properties of sequence variants are applicable to non-coding variants.
They range from single nucleotide variants to small insertion and deletion of less than $50bp$ or indels, to larger structural variants.
The latter can be copy number variants CNV or copy-number neutral.
An average human genome contains $4$ million germline sequence variants, whereas a tumour genome contains thousands of variants relative to the same individual germline DNA.
Somatic variants are rarer in healthy tissues.
Somatic mutation frequency varies across different cancer types.
Some germline variants may be responsible for tumorigenesis (high penetrance) or modulate the effect of somatic variants (low penetrance).
The germline variants associated with increased cancer susceptibility do not have a fitness effect at reproductive age, which can be the reason for the continued prevalence of such variants in the population.
Germline variants show LD that increase the difficulty in disentangling the causal disease variants.
A much higher fraction of somatic variants consist of structural variants and unlike germline variants they happen on a specific tissue.
However germline variants can have a functional effect in specific tissue if they occur in regions of closed chromatine or if they disrupt a binding site of a tissue-specific transcription factor.
Kataegis is characteristic only of somatic variants.
Moreover somatic variants are not inherited and so they are not subject to meiosis and do not show LD.

\section{Non-coding element annotation}
Non coding elements can have diverse roles in the regulation of protein-coding genes.
They consist of cis-regulatory regions and ncRNAs.
They are identified by functional genomics approaches or sequence conservation and display cell and tissue specificity.

	\subsection{Cis regulatory regions}
	Cis-regulatory regions include promoters and distal elements which regulate gene expression following binding by TFs.
	TFs bind to specific DA sequences within thier larger regions of occupancy which can be identified using chromatin immunoprecipitation followed by sequencing assays.
	They bind DNA in regions of open chromatin identified using DNAase I hypersensitivity assays and DNAase I footprinting.
	DNA methylation and other histone modification can modulate TF accessibility.
	Several histone marks are ssociated with specific putative functions.
	Most of sequence-specific TF and chromatin maks lead to highly localized ChIP-seq signals, other marks are associated with large genomic domains.
	Epigenetic hcanges can alter FT accessibility in  different cellular states and can change the activity or regulatory elements.


	\subsection{Distal regulatory elements}
	Distal regulatory elements may regulate gene expression by interacting with promoters in the 3D structure of the genome.
	Linking them to their target region is crucial to understand the effects of sequence variants in them.
	Multiple approaches have been used linke chromosome conformation capture: regulatory sequences can control transcription by looping to and physically contacting target coding genes that are located tens or hundreds of kilobases away.
	It probes one-versus-one contacts in the 3D space of the genome.
	Other variation control one-versus-all, many-versus-many and all-versus-all contacts.
	Other approches include correlation of histone marks at enhancer regions and target gene expression across multiple cell lines.
	Links between epression quantitative trait loci and associated genes.
	The resulting linkages can be studied as a comprehensive network.

	\subsection{RNA-seq}
	RNA-seq reveals non-coding transcripts, which can be confirmed to not have protein-coding ability by the absence of open reading frames or proteomic analysis.
	Certain histone modification can also indicate ncRNA activity.
	ncRNA can be divided into categories and they act through different mechanisms to modulate genexpressions.
	In particular miRNA and lncRNA are important in cancer biology.
	miRNA inhibit target gene expression by binding to the 3'-UTR and causing mRNA degradation or repression of translations.
	The mechanisms of action of lncRNA remain unclear, but a number of lncRNA have been shown to act as molecular scaffolds that bind proteins, DNA or other RNA molecules and are able to modulate gene expression.

	\subsection{Transcribed pseudogenes}
	Transcribed pseudogenes are a type of ncRNA tha bear a clear resemblance to functioning protein coding genes.
	They are copies of coding genes that have lost their ability to code for proteins owind to disabling mutations.
	They can be divided into duplicated and processed based on their formation from duplication or retrotransposition of the parent gene.
	Processed pseudogenes lack the promoter sequence and intronic strucutre and contain a 3'-poly(A) tail.
	These pseudogenes can be transcribed and regulated the expression of their parent genes, generating endo-siRNA and partecipating in the RNA interference pathway or by acting as molecular sponges.

	\subsection{Evolutionary conservation}
	Evolutionary conservation of genomic sequence across multiple species is used to annotate non-coding regions.
	Comparative analysis allowed the discovery of these ultra-conserved elements, the majority of which do not overlap protein-coding exons.
	Analysis of these sequence is important because they have been show to have a role in cancer biology.
	Non-coding elements exhibit conservation among humans.
	Negative selection within the population can be estimated using enrichment of rare alleles and reduced density of single nucleotide polymorphisms.
	These can be important to identify elements that show human-specific conservation in functional non-coding categories.
	The ultra sensitive elements and have strong depletion of common polymorphisms and enrichment of known disease-causing mutation
	Negative selection can be used to identify candidate cancer driving mutations.

\section{Roles for somatic variants in cancer}
