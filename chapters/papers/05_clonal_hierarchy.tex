\chapter{Unravelling the clonal hierarchy of somatic genomic aberrations}

\section{Introduction}

	\subsection{Abstract}
	Defining the chronology of molecular alterations may identify milestones in carcinogenesis.
	The analyses highlight the diversity of clonal evolution within and across tumour types that might be informative for risk stratification and patient selection for targeted therapies.

	\subsection{Background}
	Cancer arises from clones that undergo intense evolutionary selection during disease progression.
	THis process may lead to subclonal divergence resulting in genetic and molecular heterogeneity.
	Several methods have been developed to quantify DNA admixture and ploidy from SNP array data that use the relative abundance of specific allele signal ($B$ allele frequency) and the tumour over normal signal ratio or Log R to measure the complexity of the cellular population.
	Using gerline heterozyfous SNP loci or informative SNPs tumour purity and ploidy are estimated analyzing allelic fraction values.
	Subclonal alterations will appear as outliers from the computed admixture and ploidy.
	Global methods are well-suited fro tumour samples with homogenous genomic aberrations.
	This approaches are suboptimal with tumour samples with high heterogeneity.
	Local optimization uses creates estimates of purity and ploidy from few clonal events.
	The AF values of informative SNPs in a somatic deletion result from the composition of signal from non-tumour cells, tumour cells without the deletion and tumour cells harbouring the deletion.
	Modelling the probability distribution of the observerd AF, a local estimate of the DNA admixture is computed, accounting for both normal cell admixture and subclonal tumour cell population.
	After the estimation for all deletions across the genome only selected regions contribute to the computation of the tumour sample global admixture.

\section{Results}

	\subsection{Clonality assessment of aberrations from sequencing reads}
	The reads mapped into a genomic window can be partitioned into a set containing reads that equally represent parental chromosomes and a set containing reads from only one parent chromosome.
	There are four steps that from neutral read counts, allow inference of clonality of any genomic window.
	First the percentage of neutral reads within a genomic segment are estimated independently of its Log R value.
	Then the Log R value is used to relate the neutral reads with a local estimate of DNA admixture.
	Local estimates are aggregated to estimate global admixture and clonality of somatic copy number aberrations.
	Aneuploidy genomes are identified and the analysis corrected accordingly.
	The analysis  is then extended to point mutations and structural rearrangements.
	For each genomic semgent $Seg$ the expected $AF$ of the informative SNP has a bimodal distribution that relates to the composition of the DNA sample.
	The distance between the two modes is proportional to the percentage of neutral reads $\beta$.
	The expected distribution of the $AF$ varies accordingly with $\beta$ and $N_{ref}$, the proportion of reference base reads in the allele represented by active reads.
	For each input segment $Seg$, optimization based on swarm intelligence finds a $\beta$ that minimizes the difference between the expected and the observed $AF$ distribution.
	Then the Log R of $Seg$ allows computing a local estimate of the admixture.
	If $Seg$ defines a mono-allelic deletion, $\beta$ corresponds to the percentage of reads deriving from cells that do not Harbor the deletion and relates to a local estimate of the percentage of admixed cells:

	$$Adm.local = \frac{\beta}{2-\beta}$$

	Local admixture values are clustered and the lowest median determines the global admixture of the sample.
	The more the local admixture value differs from the gloval the more $Seg$ is subclonal.
	The clonality of $Seg$ or $Cl_{Seg}$ is computed as the percentage of tumour cells in a sample harbouring $Seg$.
	If $Seg$ is a gain $Adm.local$ extends by rescaling the percentage of neutral reads $\beta$ to recover the percentage of reads sequenced from cell that does not harbour the gain of $Seg$.
	Bi-allelic deletions are treated separately.
	If the deletion is clonal its $AF$ has binomial distribution $\beta=1$ and represents DNA admixture.
	In case of subclonality $\beta$ is proportional to the percentage of tumour cells that do not harbour the deletion.
	Aneuplidy causes a shift in the Log R vs $\beta$ space.
	In any segment with an empty active reads set each allele has the same number of copies and $\beta=1$.
	The ploidy of a sample is the shift in the Log R values of the neutral segment that best accounts for the observed Log R values.
	Log R data are corrected for ploidy and Adm.global to achieve better estimates of the segment copy number.
	Clonality estimates build on the assumption that reads supporting the alternative allele are representative of the amount of tumour DNA harbouring the mutation.
	The proportion of reads supporting the alternative allele of a pure and clonal hemizygous PM has symmetric binomial distribution.
	Adm.global represents the percentage of reads from admixed cells that have to be ignored to compute the correct value of AP.
	A PM is subclonal when its corrected AP has a low probability to be clonal.
	The same principle applies to REARRs.
	The total numberof reads that span both sides of a breakpoint defining a REARR is a proxy of the number of cells harbouring the rearrangement.
	The difference between the expected and observed proportion of reads supporting the alternative allele is proportional  to the subclonality of the considered REARR.

	\subsection{Inferring the order of mutations in a tumour sample}
	The assessment of the clonality of each somatic aberration enables the deconvolution of the sequence of oncogenic events that occur during tumour initiation and progression.
	Assuming that clonal alterations pre-dates subclonal alterations within the same tumour, pairs of genes aberrant in the sample sample and across multiple tumours are considered to determine the directionality of the clonal-subclonal hierarchy.
	To minimize the number of of false posistives (clonal called subclonal) the estimation uncertaninty around $\beta$ is computed and propagated to clonality values.
	This enables robust comparison of aberration clonality across different tumour sample data.
	If a clonal aberration $A_1$ and a subclonal $A_2$ occur within the same sample $S$, $A_1$ has been acquired before $A_2$ in $S$ and $A_1$ precedes $A_2$ in $S$.
	The same dependency has to be found consistently across samples to derive the rule that links $A_1$ and $A_2$.
	This can produce an evolution path draft and in the presence of adequate sample size and frequencies of co-occurring aberrations, the statistical significance of the relation between $A_1$ and $A_2$ can be assessed by testing the null hypothesis that the two aberrations are independent and consider a binomial distribution  with number of trials $n$ equals the number of samples where $A_1$ is clonal and $A_2$ is subclonal or vice versa.
	With

	\subsection{In silico and in situ experimental validation}
	To assess if the coverage depth typical for large scale sequencing experiments has an effect on clonality estimates miSeq ultra-deep sequencing data was queried.
	Excellen agreement in downstram clonality calls for deletion was observed.
	CLONET did not assign clonality values to aberrations in which MiSeq does not confirm AP values.
	Next studying PMs and assessing high correlation of AP values between WGS adn MiSeq data, suggesting that the study coverage duse not significantly impact the ability to assess aberration clonality.
	In order to validate the clinality statuse of complex structural genomic aberrations, in situ tests was used.
	The ability to assess rearrangemnt clonality was deminstrated focusing on well-characterized REARRs.
	Perfect agreement was demistrantion.
	Also subclonal bi-allelic deletion was validated by fluorescence.
	The prediction highlights a small subclonal bi-allelic deletion within a larger clonal mono-allelic deletion, suggesting that selective evolutionary pressure is acting on the genomic region.

	\subsection{Comparative analysis reveals different mechanisms of tumour deregulation}
	THe mean number of events classified as clonal or subclonal by means of the proportion test with FDR correction.
	Deletions are more heterogeneous than gains in prostate and lung cancer, while melanoma had the opposite behaviour.
	Comparing the proportion of clonal/subclonal losses and gains the prostate and lung samples are statistically indistinguishable.
	This suggests that temporally distinct mechanisms lead to loss and gain across the three tumour types.
	Prostate cancer in terms of PMs exhibits more subclonal events than melanoma, suggesting a more central role of PMs in melanoma oncogenesis compared with prostate cancer.
	Aggregated values reflect only part of the story: great variability in the percentage of clonal events within a single combination of tumour and aberration is observed.
	Then the distribution along the genome of the variability in the clonality status of aberrations was assessed.
	Commonality beween the three tumour types in some regions can be observed.
	Then the capability of clonality analysis to highlight tumour specific mechanism of deregulation was investigated.
	Considering PTEN deletion, which is involved in many cancer types, it was seen how the timing of the alteration is different and may point to differential roles for pathway inactivation.
	The focal and subclonal deletion in prostate samples suggests that evolutionary pressure is acting later and may promote cancer progression at a later stage.
	PTEN is homogenously lost in metastatic melanoma.
	In lung cancer this loss is more rare.
	CLONET can identify tumour lineage specific subclonality.

	\subsection{Clonal hierarchy of genomic aberrations}
	The temporal evolution of driver aberrations was analysed to build evolution maps capitalizing on the information from multiple individuals' samples in the absence of multiregion samples.
	Given the sample size and the mutation frequencies, drafts of evolution maps were built by implementing the following rule.
	In particular, an arrow from $A_1$ to $A_2$ is drawn if:

	\begin{multicols}{2}
		\begin{itemize}
			\item $A_1$ and $A_2$ co-occur in at least two samples.
			\item $A_1$ preceded $A_2$ in at least one sample.
			\item $A_2$ does not precede $A_1$ in the considered dataset.
		\end{itemize}
	\end{multicols}

	The sensitivity of CLONET allowed the identification of additional genes whose loss precedes the homozygous deletion.
	No contradictory relations were detected in independent datasets.
	In order to investigate common patterns of progression across tumour types, a large set of putative cancer genes was interrogated and applied pairwise intersections of identified paths.
	The evolution of known cancer signalling pathways was explored: both ommon themes across tumour types and tissue-specific patterns emerged.
	Recurrently deregulated pathways where detected as early drivers.
	THe timing of dysregulation along the evolutionary paths can be independent across tumour types.

\section{Materials and methods}

	\subsection{CLONET pipeline}
	SNPs have been extracted from BAM files using an in-hous procedure, SCNAs were detected using SegSeq from tumour and normal sequencing-based data, PM coordinates were as in original corresponding manuscripts and REARRs were identified by means of dRanger and Breakpointer.
	To avoid germline background effects, genes that intersect significant with known germline copy number variants were filtered out.

	\subsection{CLONET on exome and targeted sequencing data}
	The analysis of samples with few SCNAs provided that informative SNPs read counts and Log R values are available is enabled.
	Individual specific informative SNPs can be identified from matched normal DNA samples.
	Appropriate Log R values can be optained for exome genomi segments with platform specific strategies and provided to CLONET as input.
	Array-based segemnted data or SCNA segments directly inferred from eome data with recent well-performing tools.
	CLONET combines segment input with exome-derived read counts to estimate purity and ploidy.
	Then subclonal aberrations are called based on sequencing data.
	Copy number calls derived using custom control regions and very hig-coverage allowed for CLONET based clonality estimation even in the case of low tumour content.

	\subsection{Expected distribution of the allelic fraction of a genomic segment}
	Consider a genomic segment that spans a set of informative SNPs for the individual of intersect.
	For any of them with coverage $cov$ the total number of reads $r$ supporting the reference base is the sum of the neutral reads $r_n$ and the active reads $r_a$ supporting the reference base.
	$\beta$ is the ration between neutral reads and the total number of reads spanning the SNP of interest.
	The probability of having $k$ reference reads is the convolution of the probability of observing $\beta\cdot k$ neutral reads and $(1-\beta)\cdot k$ active reads:

	$$P(r = k, 0\le k\le cov) = Conv(P(r_n = \beta\cdot k), P(r_a = (1-\beta)\cdot k))$$

	$P(r_n = \beta\cdot k)$ is assumed to follow a binomial distribution with trials $\beta\cdot cov$ and probability of success $ps$.
	All active reads support the reference or the alternative base.
	$N_{ref}$ is the proportion of informative SNPs within the aberration that carry the SNP reference base in the allele represented by the active reads.
	$P(r_a = (1-\beta)\cdot k)$ follows a categorical distribution iwth values equal to $N_{ref}$.

	$$P(r = k|cov, \beta, N_{ref}, ps) = (1-N_{ref})\cdot B(k|\beta\cdot cov, ps) + N_{ref}\cdot B(k-(1-\beta)\cdot cov | \beta\cdot cov, ps)$$

	Where $B(m|n, p)$ is the probability mass function of a binomial distribution, the probability of $m$ successes in $n$ trials with success probability $P$.

	\subsection{Estimated proportion of neutral reads for a genomic segment}
	$\beta$ and $N_{ref}$ can be inferred from the sequencing coverage at informative SNPs within the segment.
	Given a segment $Seg$ and a set $I$ of informative SNPs in $Seg$, each SNP in $I$ is a sample from the distribution described earlier.
	Optimization can allow for the identification of values $\beta$ and $N_{ref}$ for each segment using the Kolmogorov-Smirnov for the likelihood that $I$ are a sample of the distribution and a particle swarm optimization finds a candidate $\hat{\beta}$ and $\hat{N}_{ref}$ that best represents the distribution of the allelic fraction of the SNPs in $I$.

	\subsection{From neutral to non-aberrant reads}
	Consider a $Seg$ if the $Log\ R$ value of $Seg$ support a SCNA $C$, reads that cover $Seg$ from cells harbouring $C$ are considered aberrant.
	If $Seg$ is a candidate mono-allelic deletion $\beta$ corresponds to the percentage of reads that cover $Seg$ and are sequenced from cells harbouring both alleles.
	If the Log R value supports a gain with $cn >2$, $\beta$ has to be rescaled to obtain the percentage of sequenced cells that have copy number $cn$.
	If $cn$ is odd, the number of neutral reads is the sum of the neutral from admixed plus the neutral of the gain.
	$\beta_{cn}$ of reads from cells with $cn$ is computed from $\beta$ by removing neutral reads due to the gain:

	$$\beta_{cn} = 1 - cn_{G}\cdot (1-\beta)$$

	If $cn$ is even and one copy difference between allele is allowed $\beta$ is closed to one.

	\subsection{From aberrant reads to aberrant cells}
	Given a somatic mono-allelic deletion $M$ the local admixture $Adm.local$ is the proportion of cells not harbouring $M$ over the total number of cells.
	Let $a$ define the total number of reads supporting the alternative allele,, as the sum of neutral $a_n$ and active $a_a$ reads.
	For any informative SNP within $M$, the local admixture is:

	$$Adm.local_M = \frac{\frac{r_n+a_n}{2}}{\frac{r_n+a_n}{2}+(r_a+a_a)}$$

	The proportion of non-aberrant reads covering $M$ is:

	$$\beta_M = \frac{r_n+a_n}{r_n+a_n+r_a+a_a}$$

	\subsection{Uncertainty assessment and its propagation to clonality estimates}
	To optimize sensitivity and specificity the estimation uncertainty $\epsilon$ around $\beta$ is computed.
	The value of $\epsilon$ varies upon the mean coverage and the number of informative SNPs.
	The mean coverage controls the ability to discern the two modes of the AF distribution.
	Higher $\beta$ requires higher coverage.
	The procedure to infer the value of $\beta$ is independent from its Log R value.
	Segments aggregate into cluster corresponding to copy number and define a clonality status.
	Restricting to putative somatic mono-allelic deletions, $B_{min}$ with the lowest median of $\beta$ would represent $100\%$ clonal deletions.
	$B$ is the set of $\beta$ values of all the putative somatic mono-allelic deletions.
	$B_{min}$ is the smallest subset of $B$ such that $min(B)$ in $B_{min}$ and for all $\beta'$ in $B$ and not not in $B_{min}$, $max(B_{min})+err(max(B_{min})) < \beta'-err(\beta')$.
	The median value is selected as candidate $Adm.global$.
	Given a somatic copy number $C$ in a sample, the local and global admixtures are computed.
	The clonality $Cl_C$ of $C$ is the percentage of timor cells in a sample harbouring $C$:

	$$Cl_C = \frac{1-Adm.local_C}{1-Adm.global}$$

	The more the value local differ from the global the more $C$ is subclonal.

	\subsection{Clonality of bi-allelic deletion}
	For subclonal bi-allelic deletion the allelic fraction signal comes from cells with two or one allele.
	Consider a subclonal bi-allelic deletion where $n$, $m$ and $b$ denote the proportion of cells with two, one and zero alleles.
	The local estimate of the admixture can be computed.
	This is the proportion of cells with two alleles in the subpopulation of cells with one or two alleles, $n = Adm.local(n+m)$.
	The proportion of normal cells in the sum is equal to the global DNA admixture.
	The clonality of a bi-allelic deletion $Cl_B$ is the percentage of cells harbouring the bi-allelic deletion over the number of cells with a mono- or a bi-allelic deletion $\frac{b}{m+b}$.

	$$Cl_B = \frac{Adm.global - Adm.local\cdot Adm.global}{Adm.local\cdot(1-Adm.global)}$$
